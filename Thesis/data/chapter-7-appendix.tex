% 附页\emph{}
\chapter{攻读硕士学位期间取得的学术成果}
% 此处标题及内容请自行更改
\noindent 发表论文:
\begin{enumerate}[label=\arabic*.]
\item \textbf{Nan Jiang}, Wenge Rong, Min Gao, Yikang Shen and Zhang Xiong. Exploration of Treebased Hierarchical Softmax for Recurrent Language Models[A]. Proceedings of the 26th International Joint Conference on Artificial Intelligence[C]. 2017: 1951-1957.
\item \textbf{Nan Jiang}, Wenge Rong, Yifan Nie, Yikang Shen and Zhang Xiong. Event Trigger Identification with Noise Contrastive Estimation[J]. IEEE/ACM Transactions on Computational Biology and Bioinformatics, 2017. (Accepted)
\item \textbf{Nan Jiang}, Wenge Rong, Baolin Peng, Yifan Nie and Zhang Xiong. Modeling Joint Representation with Tri-Modal DBNs for Query and Question Matching[J]. IEICE Transactions on Information and Systems, 2016, 99(4): 927-935.
\item \textbf{Nan Jiang}, Wenge Rong, Baolin Peng, Yifan Nie and Zhang Xiong. An Empirical Analysis of Different Sparse Penalties for Autoencoder in Unsupervised Feature Learning[A]. Proceedings of 2015 International Joint Conference on Neural Networks[C]. 2015: 1-8.
\end{enumerate}
\noindent 投稿论文:
\begin{enumerate}[label=\arabic*]
\item \textbf{Nan Jiang}, Wenge Rong, Min Gao, Yikang Shen and Zhang Xiong. Exploration of Hierarchical Softmax for Recurrent Language Models[J]. ACM Transactions on Intelligent Systems and Technology. (Submitted)
\end{enumerate}
\chapter{致\quad 谢}
在2015年,我来到了北航计算机学院工程研究中心,开始了我研究生学习阶段。首先感谢荣文戈副教授当年的知遇之恩,没有导师指点与意见,我也不会成为现在的我。同时,更想感谢熊璋老师、欧阳元新老师和王静远老师,他们对我的谆谆教导,是我在遇到苦难的时候能够不言放弃,努力解决问题所在。

我能够顺利完成研究生阶段的求学,离不开荣老师的悉心指导和耐心改正我的错误。荣老师不仅为我设定了远大的目标,还真且关注我的水平的成长,每次都是设置一个能力可及的任务,不断锻炼我的写作技能和实验技能。最令人记忆深刻的是,在每次论文投稿前一周,每当深夜我将论文修订完一版本之后,荣老师总是立马修改,甚至在早上四点给我修改论文。顶着巨大的身体压力和时间压力,给我不断修改论文,还不断跟我捋顺论文思路,其耐心已经超越了我人生认识的所有老师。然而学术的生涯并不总是一帆风顺的,在很长一段时间的瓶颈期,荣老师在听完我不断的抱怨之后,仍然鼓励我,帮我树立做科研的自信心,使我明白不断加强自身知识和技术水平的重要性。经过整整大半年的低谷时期,总算迎来了一点新的成果,此时的我非常膨胀,自视甚高。此时荣老师又劝导我,过度自信和过度自我否定都是不可取的,人生路尚且长,我们需要走好每一步,不要因为别人的质疑而否定自己,更不能因为别人的赞美而吹嘘自己。

%自小离家求学,父母的管教未曾领受,导致现在的我存在诸多性格和学习态度上的缺点,荣老师在这两年里,如醍醐灌顶一般,将他的人生哲学授予我,实在是不可多得的财富。相比于那些物质上的利益,导师传授的这些哲学才是颠扑不破的道理。

还要感谢实验室的学长和学弟对我的呵护和指点。其中包括:陈虞君博士、张硕师姐、聂一凡大师兄、沈驿康师兄。还有已经毕业的宋欣、谢维柱师兄。令人记忆深刻的是我当初投稿阶段,实验需要很多台计算设备,他们听到我的需求,帮助我在他们的高精度计算平台上开展实验工作。除此之外,我还要感谢可爱的师弟师妹们,是黎彬、田川师弟。还要感谢邱晨和高志峰提供源源不断有关实验和算法的建议。

最后感谢我的父母,由于家境不是很好,父母从我三岁开始去上海打工,历时20余载,一直坚守在自己的岗位上,为了我的以后的未来,积攒下一点积蓄。尽管他们远在他乡,对我的学业也是非常的关心,每周都得电话通讯,报告进期的教授的知识和学业成绩。没有一句累了不想干了,我因有他们一直陪在我身边感到幸运,因他们的不懈付出而感到学业有成的重要性。