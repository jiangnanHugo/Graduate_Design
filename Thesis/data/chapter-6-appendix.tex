% 附页\emph{}
\chapter{攻读硕士学位期间取得的学术成果}
% 此处标题及内容请自行更改
\noindent 发表论文:

\noindent 1. \textbf{Nan Jiang}, Wenge Rong, Min Gao, Yikang Shen and Zhang Xiong. Exploration of Tree-based Hierarchical Softmax for Recurrent Language Models[C]. Proceedings of the Twenty-Sixth International Joint Conference on Artificial Intelligence (IJCAI), 2017, pp. 1951-1957. (已发表)

\noindent 2. Yikang Shen, Wenge Rong, \textbf{Nan Jiang}, Baolin Peng, Jie Tang and Zhang Xiong. Word Embedding Based Correlation Model for Question/Answer Matching[C]. Proceedings of the Thirtieth {AAAI} Conference on Artificial Intelligence (AAAI), 2017, pp. 3511-3517.(已发表)

\noindent 3. \textbf{Nan Jiang}, Wenge Rong, Yifan Nie, Yikang Shen and Zhang Xiong. Event Trigger Identification with Noise Contrastive Estimation[J]. IEEE/ACM Transactions on Computational Biology and Bioinformatics, 2017, pp. 1-11.(已发表)

\noindent 4. \textbf{Nan Jiang}, Wenge Rong, Baolin Peng, Yifan Nie and Zhang Xiong. Modeling Joint Representation with Tri-Modal DBNs for Query and Question Matching[J]. IEICE Transactions on Information and Systems, 2016, 99(4): 927-935.(已发表)

\noindent 5. \textbf{Nan Jiang}, Wenge Rong, Baolin Peng, Yifan Nie and Zhang Xiong. An Empirical Analysis of Different Sparse Penalties
for Autoencoder in Unsupervised Feature Learning[C]. International Joint Conference on Neural Networks (IJCNN), 2015, pp. 1-8.(已发表)

\chapter{致\quad 谢}
在2015年,我来到了北京航空航天大学计算机学院先进计算机技术教育工程研究中心,开始了我为期三年的研究生学习阶段。
作者要感谢是立彬,邱晨和高志峰提供了有关实验和算法的建议。