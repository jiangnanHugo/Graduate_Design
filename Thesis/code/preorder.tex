
\documentclass[master,openright,oneside,color]{../buaathesis}
\usepackage{booktabs}
\usepackage{courier}
\usepackage{minted}
\usemintedstyle{pastie}
\setmonofont{Courier}
\renewcommand{\theFancyVerbLine}{\sffamily
\textcolor[rgb]{0.5,0.5,1.0}{\scriptsize
\oldstylenums{\arabic{FancyVerbLine}}}}
\begin{document}

\begin{minted}[mathescape,linenos=False,gobble=1,frame=lines,baselinestretch=1.0,framesep=1mm]{python}
 def pre(self,path=None,param=None,words=None):
   # 初始化
   if words is None: words=[]
   if path is None:  path=[]
   if param is None: param=[]

   if self.left: # 左子树存在
     if isinstance(self.left[1],Node):
       self.left[1].pre(path+[-1],param+[self.index],words)
     else: # 访问到左叶子节点
       word=(self.left[1],param+[self.index],path+[-1])
       words.append(word)
   if self.right: # 右子树存在
     if isinstance(self.right[1],Node):
       self.right[1].pre(path+[1],param+[self.index],words)
     else:  # 访问到右叶子节点
       word=(self.right[1],param+[self.index],path+[1])
       words.append(word)
   return words #单词路径查找表 Γ
\end{minted}
\end{document}
