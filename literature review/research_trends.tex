\section{研究趋势}
虽然现在基于物理的流体模拟的数学基础是CFD,但由于计算机图形学中更关心的
是渲染的效率以及最终结果在视觉上的真实程度,因而计算机图形学中对流体的
仿真和CFD的侧重点不同。对于计算机动画来说,希望能够在保留更多细节特征的
前提下找到一种快速、稳定的求解算法,医保证人们能够在交互设计的过程中就
能够直接观察到效果并进行调整。同时人们还期望能够对火焰的流体特征进行更
充分的控制,使其最终的画面效果能够满足人们的意愿,而不需要完全遵从物理
规律。

\subsection{流体动画控制}
为了满足在实际应用中需要对火焰的燃烧进行控制的需求,一个实用的流体仿真
算法还必须提供灵活的火焰动画的控制。对于基于NS方程的流体的模拟,可以通
过对方程中的各个参数的调整来达到控制火焰动画的目的。可以控制的参数包括
流体本身的属性,如粘性系数、热膨胀系数,还包括边界条件、障碍物、初始条
件等等。但在很计算机图形学的多应用领域中并不关心模拟结果和物理规律是否
吻合,要求在达到很好视觉效果的前提下对流体动画进行人为的控制。如何满足
实际应用中的不同需求,对火焰的流体的动画提供更灵活的控制方式是进后的一
个研究方向。

\subsection{实时绘制}
在现在的电影、科学仿真等实际应用中通过使用离线计算已经可以达到仿真度相
当高的火焰的渲染。但是在游戏、模拟训练等需要即时渲染的应用中仍然还在使
用基于粒子系统的火焰仿真,其中一个重要原因就是基于物理的模拟方法的计算
量太大。随着硬件的不断发展使得一些算法的即时计算成为了可能,研究者在追求
真实感的同时也在努力的提升计算速度。

随着近年来具有强大并行计算能力的GPU的出现,很多研究者转向通过硬件加速渲
染的方法。Mark J. Harris\cite{Harris}在GPU上实现了一个快速稳定的二维流
体模拟方法。而柳有权\cite{Liuyouquan}则利用GPU的并行计算能力强的特点来
解NS方程组,实现了一种基于GPU的带有复杂边界的实时三维流体的模拟。已经有
很多文献实现了特定精度和场景范围下的基于物理火焰模拟的即时渲染,但是运
行速度仍然达不到实际应用的要求。随着nVidia CUDA,OpenCL,AMD Stream等
技术的发展,利用GPU来对火焰的模拟进行提速也会成为一个发展的方向。

\subsubsection{细节表现}
火焰模拟时对真实感的一大要求是火焰的细节,对细节的加强是人们对真实感的
要求。由于火焰不稳定的特性导致即使是稳定燃烧的火焰其外观会随着时间快速
的变化。细节的丢失会会导致严重的失真。当使用基于物理的方法来模拟火焰的
时候细节主要取决于选取的方程以及方程的求解精度。随着在流体模拟上的研究
的不断发展,对于流体细节表现的关注会越来越多。对流体细节更真实的突出能
够更好的体现出整个流体的真实感,而对细节本身的模拟又来自与对物理模型更
精细、准确的描述和求解。

\subsubsection{流体与场景的交互}
流体与场景的交互,即流固耦合问题,就是额外考虑流体和固体之间的相互作
用,如水中漂浮的物体、风中飞扬的旗帜等。具体到火焰的模拟中,就是火焰燃
烧导致的物体的形变、解体,以及燃烧中的物体反过来导致的火焰的变化。

最初的很多文献只处理了单项作用,如固体对流体的作用,即固体的运动已
知,利用这个固体来改变流体的运动状态,而固体本身不受流体影响;或者流体
对固体的作用,即固体在流体的作用下发生运动,而固体的运动却不影响流体。
显然只靠率单项的作用并不足以表现真实的运动,所以很多研究开始转向流固耦
合,如文献\cite{burning_objects03}\cite{burning_objects05}等。